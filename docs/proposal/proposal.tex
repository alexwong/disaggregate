\documentclass{article}
\usepackage[utf8]{inputenc}
\usepackage[margin=1.25in]{geometry}
\usepackage{amsmath}
\usepackage{amssymb}
\usepackage{stmaryrd}
\usepackage{listings}
\usepackage{setspace}
\usepackage{graphicx}
\usepackage{hyperref}
\usepackage{selectp}
\setlength\parindent{0pt}
\let\ACMmaketitle=\maketitle
\renewcommand{\maketitle}{\begingroup\let\footnote=\thanks\ACMmaketitle\endgroup}

\lstset{
  basicstyle=\ttfamily,
  columns=fullflexible,
}

\title{Proposal for Resource Interferance Characterization on Disaggregated Datacenters}
\author{
    \vspace{6px}
    Alex Wong, Juexiao Wang, Junyi Huang \\
    Department of Electrical and Computer Engineering, Cornell University \\
    \texttt{\{aw528\}@cornell.edu}
}

\begin{document}
\maketitle

\onehalfspacing

\section{Introduction}
Disaggregated systems are a new approach to datacenters which allow scaling up at a resource granularity. Rather than just upgrading or adding individual servers (which contain a CPU, network, and memory) we can individually scale up the number of CPUs, memory, or network capabilities by itself. However, as you scale up, you will have an increasing number of applications which will cause interference throughout the entire system due to additional pressure on the network or memory. The goal of this project is to attempt to characterize this relation (how scaling up resources affects the resource interference) as well as different methods to isolate resources.

\section{Proposal}

\section{Conclusion}

\end{document}
