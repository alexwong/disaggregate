\documentclass{article}
\usepackage[utf8]{inputenc}
\usepackage[margin=1.25in]{geometry}
\usepackage{amsmath}
\usepackage{amssymb}
\usepackage{stmaryrd}
\usepackage{listings}
\usepackage{setspace}
\usepackage{graphicx}
\usepackage{hyperref}
\usepackage{selectp}
\setlength\parindent{0pt}
\let\ACMmaketitle=\maketitle
\renewcommand{\maketitle}{\begingroup\let\footnote=\thanks\ACMmaketitle\endgroup}

\lstset{
  basicstyle=\ttfamily,
  columns=fullflexible,
}

\title{Proposal for Resource Interferance Characterization on Disaggregated Datacenters}
\author{
    \vspace{6px}
    Alex Wong, Juexiao Wang, Junyi Huang \\
    Department of Electrical and Computer Engineering, Cornell University \\
    \texttt{\{aw528, jw2534, jh2635\}@cornell.edu}
}

\begin{document}
\maketitle

\onehalfspacing

\section{Introduction}
Disaggregated systems are a new approach to datacenters which allow scaling up at a resource granularity. Rather than just upgrading or adding individual servers (which contain a CPU, network, and memory) we can individually scale up the number of CPUs, memory, or network capabilities by itself. However, as you scale up, you will have a potentially increasing number of applications which will cause interference throughout the entire system due to additional pressure on the network or memory. The goal of this project is to attempt to characterize this relation (how scaling up resources affects the resource interference) as well as different methods to isolate resources.

\section{Simulator}
In order to characterize how these applications affect the system, we will build an emulator for the simplest representation of a disaggregate system (although the actual design will change depending on what isolation mechanisms we want to test). The naive solution is to model a very simple CPU architecture and have multiple CPUs running as different threads in our emulator. We'll define the different memory blocks as some global memory along with correspond APIs for reading/writing that allow locked access to the memory between the different CPUs. The unified interconnect between the threads and the different memory blocks can just be inferred based off the implementation of the memory (in that every thread (CPU) has access to the memory, but we only have 1 port for r/w) and a counter where we can define the total bandwidth capacity of the network. We can also implement more complicated network topologies via some virtual mapping to the actual memory blocks. Similar to an actual dissaggregate system, we will allow scaling up at the resource granularity (ie: increase the number of CPUs doesn't affect the amount of memory). We haven't decided on what architecture to have the CPU implementation map to, but we will leverage gcc so we don't have to write a compiler to specifically target the emulator.

\section{Applications}
In order to benchmark the performance of the disaggregated system, we plan to deploy several applications over the system. As we scale up the system and introduce more applications, there will be more sharing between the different resources and consequently, interference. To benchmark the interference over different kind of resources, we plan to deploy multiple applications that will cause interference with regard to different source of resources. To benchmark the interference on shared memory, memory-intensive application such as matrix multiplication is a good choice. The most naive implementation of matrix multiplication is a for-loop where we fetch memory, perform some computation, and then store back in some memory. In other words, this type of application will showcase both a lot of network traffic as well as memory accesses. Another way to benchmark the interference on network bandwidth is by having the application send many requests to the system. As we are simulating the disaggregated system, we can represent the frequent request with the frequent execution of the program running the application. We'll support different applications which invoke different kinds of resource interference and  benchmark different characteristics of the design of our disaggregated system.

\section{Isolation Mechanisms}
There are some popular isolation mechanisms used in datacenters to reduce resource contention. Three isolation mechanisms will plan to experiment with in this project are containers, cache partition and network bandwidth partitioning. Containers package up code and all dependencies so that an application can quickly and reliably run between different compute environments. By default, containers typically have no resource constraints but they can also implement different resource control mechanisms. Cache partitioning is another common resource isolation mechanism by changing the portion of a shared CPU cache available to individual tasks. A common form of cache partitioning is way partitioning, which realizing by allocating subsets of cache ways to processes. Another isolation mechanism, network partitioning enables administrators to split up the network pipe to divide and reallocate bandwidth and resources as needed. There are several approaches of network partitioning, like network-centric, user-centric, etc. The network-centric approach fairly queues the users/flows. The user-centric approach responds to user traffic according to congestion. Both the approaches could be applied in the system to see if any imrpovement in the network interference.

\end{document}
